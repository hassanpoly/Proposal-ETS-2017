\documentclass[letterpaper,12pt,oneside,final]{book}
\usepackage{bbding}
\usepackage{graphicx}
\usepackage[cmex10]{amsmath}
\usepackage{array}
\usepackage{mdwmath}
\usepackage{mdwtab}
\usepackage[caption=false,font=footnotesize]{subfig}
\usepackage{fixltx2e}
\usepackage{url}

\usepackage{booktabs}
\usepackage[capitalise]{cleveref}
\usepackage{mdwlist}
\usepackage{multirow}
\usepackage[comma,numbers,compress,square]{natbib}
\usepackage{pgfplots}
\usepackage{placeins}
\usepackage{xspace}

\usepackage{algpseudocode}
\usepackage{algorithm}
\usepackage{paralist}
\usepackage{wasysym}
\usepackage{comment}
\usepackage{color}
\usepackage{leading}
\usepackage{filecontents}
\usepackage{verbatim}
\usepackage{balance}
\usepackage{siunitx}
\usepackage{tikz}
\usepackage{pgf}
\usepackage{xxcolor}
\usepackage[colorinlistoftodos]{todonotes}
\usepackage{amssymb,amsmath}

\usepackage{pgfgantt}


\usetikzlibrary{arrows,shadows,petri,automata,shapes,shadows,trees}

\tikzset{
  basic/.style  = {draw, text width=2cm, drop shadow, font=\sffamily, rectangle},
  root/.style   = {basic, rounded corners=2pt, thin, align=center,
                   fill=green!30},
  level 2/.style = {basic, rounded corners=6pt, thin,align=center, fill=green!60,
                   text width=8em},
  level 3/.style = {basic, thin, align=left, fill=pink!60, text width=6.5em}
}
\usepgfplotslibrary{groupplots}

\hyphenation{op-tical net-works semi-conduc-tor}


%%
%%  Template de mémoire de maîtrise ou thèse de doctorat.
%%  Normalement, il n'est pas nécessaire de modifier ce document
%%  sauf pour changer les noms des fichiers à inclure.
%%
%%  Version: 2010-03-16
%%
%%  Accepte les caractères accentués dans le document (UTF-8).
\usepackage[utf8]{inputenc}

%%
%% Support pour l'anglais et le français (français par défaut).
\usepackage[cyr]{aeguill}
\usepackage[english,frenchb]{babel}
%%
%% Charge le module d'affichage graphique.
\usepackage{graphicx}
%%
%% Recherche des images dans les répertoires.
\graphicspath{{./images/}{./dia/}{./gnuplot/}}
%%
%% Un float peut apparaître seulement après sa définition, jamais avant.
\usepackage{flafter,placeins}
%%
%% Utilisation de natbib pour les citations et la bibliographie.
\usepackage{natbib}
%%
%% Autres packages.
\usepackage{amsmath,color,soulutf8,longtable,colortbl,setspace,ifthen,xspace,url,pdflscape}
%%
%% Support des acronymes.
\usepackage[nolist]{acronym}
\onehalfspacing                % Interligne 1.5.
%%
%% Définition d'un style de page avec seulement le numéro de page à
%% droite. On s'assure aussi que le style de page par défaut soit
%% d'afficher le numéro de page en haut à droite.
\usepackage{fancyhdr}
\fancypagestyle{pagenumber}{\fancyhf{}\fancyhead[R]{\thepage}}
\renewcommand\headrulewidth{0pt}
\makeatletter
\let\ps@plain=\ps@pagenumber
\makeatother
%%
%% Module qui permet la création des bookmarks dans un fichier PDF.
%\usepackage[dvipdfm]{hyperref}
\usepackage{hyperref}
\makeatletter
\providecommand*{\toclevel@compteur}{0}
\makeatother
%%
%% Définitions spécifiques au format de rédaction de Poly.
\usepackage{MemoireThese}
%%
%% Définitions spécifiques à l'étudiant.
%% -----------------------------------
%% ---> A MODIFIER PAR L'ETUDIANT <---
%% -----------------------------------
%%
%% Commandes qui affichent le titre du document, le nom de l'auteur, etc.
\newcommand\monTitre{METHODOLOGY AND ALGORITHMS FOR HIGH-LEVEL MODELLING OF COSMIC RADIATION IMPACTS ON ELECTRICAL SYSTEMS}
\newcommand\monPrenom{Hassan}
\newcommand\monNom{Anwar}
\newcommand\monDepartement{génie electrique}
\newcommand\maDiscipline{génie electrique}
\newcommand\monDiplome{D}        % (M)aîtrise ou (D)octorat
\newcommand\anneeDepot{2017}
\newcommand\moisDepot{DECEMBER}
\newcommand\monSexe{M}           % "M" ou "F"
\newcommand\PageGarde{N}         % "O" ou "N"
\newcommand\AnnexesPresentes{O}  % "O" ou "N". Indique si le document comprend des annexes.
\newcommand\mesMotsClef{keywords,separated,by,commas}
%%
%%  DEFINITION DU JURY
%%
%%  Pour la définition du jury, les macros suivantes sont definies:
%%  \PresidentJury, \DirecteurRecherche, \CoDirecteurRecherche, \MembreJury, \MembreExterneJury
%%
%%  Toutes les macros prennent 4 paramètres: Sexe (M/F), Prénom, Nom, Titres
%\newcommand\monJury{\PresidentJury{F}{Gabriela}{Nicolescu}{Ph.D.}\\

\newcommand\monJury{\DirecteurRecherche{M}{Claude}{Thiebeault}{Ph.D.}\\
\CoDirecteurRecherche{M}{Yvon}{Savaria}{Ph.D.}}
%\MembreExterneJury{F}{Jelena}{Trajkovic}{Ph.D.}}

\ifthenelse{\equal{\monDiplome}{M}}{
\newcommand\monSujet{Mémoire de maîtrise}
\newcommand\monDipl{Maîtrise ès sciences appliquées}
}{
\newcommand\monSujet{Thèse de doctorat}
\newcommand\monDipl{Philosophi\ae{} Doctor}
}
%%
%% Informations qui sont stockées dans un fichier PDF.
\hypersetup{
  pdftitle={\monTitre},
  pdfsubject={\monSujet},
  pdfauthor={\monPrenom{} \monNom},
  pdfkeywords={\mesMotsClef},
  bookmarksnumbered,
  pdfstartview={FitV},
  urlcolor=cyan
}
%%
%% Il y a un document par chapitre du mémoire.
%%
\begin{document}
%%
%% Page de titre du mémoire.
\frontmatter
% Compte optionellement la page de garde dans la pagination.
\ifthenelse{\equal{\PageGarde}{O}}{\addtocounter{page}{1}}{}
\thispagestyle{empty}%
\begin{center}%
\vspace*{\stretch{1}}
ÉCOLE DE TECHNOLOGIE SUPÉRIEURE \\
\vspace*{\stretch{1}}
\MakeUppercase{\monTitre}\\
\vspace*{\stretch{1}}
\MakeUppercase{\monPrenom~\monNom}\\
DÉPARTEMENT DE \MakeUppercase{\monDepartement}\\
ÉCOLE DE TECHNOLOGIE SUPÉRIEURE\\
\vspace*{\stretch{1}}
%\ifthenelse{\equal{\monDiplome}{M}}{MÉMOIRE PRÉSENTÉ}{THÈSE PRÉSENTÉE} EN VUE DE L'OBTENTION\\
%DU DIPLÔME DE \MakeUppercase{\monDipl}\\
\ifthenelse{\equal{\monDiplome}{M}}{whatever}{} RESEARCH PROPOSAL\\  SUBMITTED AS A PARTIAL FULFILLMENT OF THE REQUIREMENTS\\ FOR THE DEGREE OF \MakeUppercase{\monDipl}\\
(\MakeUppercase{\maDiscipline})\\
(DGA-1032)\\
\MakeUppercase{\moisDepot} \anneeDepot
\end{center}%
\vspace*{\stretch{1}}
\copyright~\monPrenom~\monNom, \anneeDepot.
%%
%% Identification des membres du jury.
%%
\newpage\thispagestyle{empty}%
\begin{center}%
\vspace*{\stretch{2}}
\ul{ÉCOLE DE TECHNOLOGIE SUPÉRIEURE}\\
\vspace*{\stretch{1}}
\ul{UNIVERSITÉ DU QUÉBEC}\\
\vspace*{\stretch{2}}
%Ce\ifthenelse{\equal{\monDiplome}{M}}{~mémoire intitulé}{tte thèse intitulée}:\\
\ifthenelse{\equal{\monDiplome}{M}}{whatever}{title of this research proposal}:\\
\vspace*{\stretch{1}}
\MakeUppercase{\monTitre}\\
\vspace*{\stretch{2}}
\end{center}%
%\begin{flushleft}
%présenté\ifthenelse{\equal{\monDiplome}{M}}{}{e}
%par:~\ul{\mbox{\MakeUppercase{\monNom} \monPrenom}}\\
%en vue de l'obtention du diplôme de:~\ul{\mbox{\monDipl}}\\
%a été dûment accepté\ifthenelse{\equal{\monDiplome}{M}}{}{e} par le jury d'examen constitué de:\end{flushleft}
\begin{flushleft}
submitted by:~\ul{\mbox{\MakeUppercase{\monNom} \monPrenom}}\\
%submitted as a partial fulfillment\\ of the requirements\\
%for the degree of:~\ul{\mbox{\monDipl}}\\
in the context of the:~\ul{\mbox{comprehensive examination}}\\
to the committee:\end{flushleft}
\vspace*{\stretch{2}}
\monJury
%%
\pagestyle{pagenumber}%
%\include{/sections/Dedicace}          % Dédicace du document.
%\include{/sections/Remerciements}     % Remerciements.
\include{./sections/Abstract}          % Résumé du sujet en anglais.
%\include{./sections/Resume_sujet}      % Résumé du sujet en français.
%%
%% Table des matières.
%\renewcommand\contentsname{TABLE DES MATIÈRES}
\renewcommand\contentsname{TABLE OF CONTENTS}
\tableofcontents
%%
%% Liste des tableaux.
%\renewcommand\listtablename{LISTE DES TABLEAUX}
%\renewcommand\listtablename{LIST OF TABLES}
%\listoftables
%%
%% Table des figures.
%\renewcommand\listfigurename{LISTE DES FIGURES}
\renewcommand\listfigurename{LIST OF FIGURES}
\listoffigures
%%
%% Liste des annexes au besoin.
%\ifthenelse{\equal{\AnnexesPresentes}{O}}{\listofappendices}{}
\include{./sections/Abbreviations}       % Liste des sigles et abréviations.
\mainmatter

\Chapter{INTRODUCTION}\label{sec:intro} \selectlanguage{english}

\section{Context and Motivation}

The research project is an EU-Canadian joint research project, which is called "EPICEA" - Electromagnetic Platform for
lightweight Integration/Installation of electrical systems in Composite Electrical Aircraft, will approach
numerous avionic engineering design issues in the advancement of future aircraft, aiming at a significant
reduction of energy consumption through more electrical aircraft and systems integration. This project strives to understand the electromagnetic (EM) issues on composite electric aircraft (CEA). This includes the analysis
and characterization of EM coupling, interconnects, and cosmic radiations (CR) on electrical systems together
with new concepts of antennas designed to maintain performance in composite environment without modifying
aircraft aerodynamics. Our contribution to this project --- "the study of CR effects on aircraft electrical systems." This research work will focus on design and implementation of the FPGA-based platform to help to investigate the effects of cosmic radiation (CR) on embedded electronic system of the aircraft. This project also aims to make the higher-level model that will use to investigate the effects of (CR) on the aircraft flying at the altitude of 40,000 feet. This project helps to find; at higher altitude when aircraft gets more exposure to the radiation; need a way to know early in the embedded electronic design of the aircraft if mitigation strategies are required to deal this higher radiation level. 

Space radiation has two preliminary sources --- galactic cosmic radiation and solar energetic particles~\cite{SWE20216}. Galactic cosmic radiation from outside the solar system consists mostly of energetic protons and heavy ions, e.g., iron. Solar energetic particles are commonly associated with the solar flare events and largely dominated by the proton. The consequences of these radiations on human health had been studied extensively at national and international levels. A global framework is also available for the addressing of these radiation issues on health particularly for the frequent flyer, e.g., air-crew. Most of the efforts done so far are either on the monitoring, modeling, and measurements of the radiation and improved the air-safety standards. The space radiation is an unavoidable space weather phenomena. The impact and consequences of the high-energy particles and thermalized neutrons on the avionics embedded system are now recognized as an area of active research. Especially, the incident happened with the Qantas Flight Airbus A330-303 flying from |Singapore to Perth went under the two terrifying dives due to the malfunction of the on-flight computer. After, the investigation it revealed that high-energy particles from the outer space --- were the responsible for the malfunction of the computer. And, the potential triggering event was the single-event effect (SEE) interacting with one of the integrated circuits (ICs) within the CPU module. 

Therefore, fault management strategies are essential to apply on the aircraft's embedded systems. In future, the FPGAs will replace the deterministic computer architecture platform provide more flexibility to flight operations. In FPGAs the configuration bits of the configuration memory that control the resources, user logic, routing resources, LUTs, CLBs, BRAM, DSP, and IOB blocks. If ion hits the FPGA, it can affect the memory resources that lead towards the fault, which may result in a failure. Before need to know to apply the mitigation techniques early in the embedded electronic design, we need to make the higher-level fault model of the systems that facilitate without going into the detailed simulation get the faulty behavior of the component at high-level.

  

\section{Problem Statement}

Cosmic rays are originating in outer space and travel at nearly the speed of light and strike the earth from all directions. These cosmic radiations are ranging from lightest to heaviest elements in the periodic table. When these high-energy cosmic rays interact with the earth's magnetosphere, neutrons are generated, often referred to as an air shower~\cite{lesea2005rosetta}. Neutron with energy greater than 10 MeV carries sufficient energy to cause single-event effects in SRAM-based FPGAs. An intense neutron environment exists at higher altitudes in the atmosphere, 10 km to 40 km.  Long-haul aircraft flying at the altitudes of 40,000 feet nearly 12 km at the latitude of \ang{60} as shown in Figure~\ref{fig:neu-flux}  under the influence of greatest neutron flux of all flights --- approximately 500 times that a ground-based observer in Newyork City~\cite{lesea2005rosetta}. This high-energy neutron passes through the silicon substrate of a device, and if the charge of these particles is sufficient enough to change the state of the configuration memory of the FPGA results in a drastic consequence. In this work, we will mainly focus on defining the pre-certification strategy that before employing the circuit in a robust condition, realistically evaluate the faulty behaviors of the circuit. The study of CR effects on aircraft at high altitude/latitude to be able to decide on the appropriate protection solution.


\begin{figure}
 \centering
  \captionsetup{justification=centering}    
   \includegraphics[scale=0.4]{figures/img/neutron-flux.png}
   \caption{Neutron Flux at 40,000 Feet.}
\label{fig:neu-flux}
\end{figure}





\section{Research Objectives}



The main objective of this project is to develop a faulty behaviour model for FPGA-based
circuits described at a high-level of abstraction.

By using neural networks, fault behaviour
models are developed and their accuracy is validated. The developed models could be used to
replace any component of the entire circuit with faulty versions of the components described
at a high-level of abstraction

The
goal of this research was to develop an approach for modeling the faulty behaviour of a
digital circuit in the presence of cosmic rays.

his ensures that the effect of faulty behaviour of each
component on a system could be analyzed at a high-level of abstraction and the mitigation
technique could be used to improve the robustness of more critical parts


The
goal of this research was to develop an approach for modeling the faulty behaviour of a
digital circuit in the presence of cosmic rays.


To implement effective high-level CR computer model one has to (a) design and implementation of the emulation system, (b) design and implementation of an experimental setup for bombardment,  c) feasible for aerospace system, and d) develop  a strategy to develop a high-level model from the results (signatures) derived from the emulation and bombardment setup. In the context of the development of a whole simulation methodology including CR environment and CR effects at
system and component levels, the objectives of this projects are:

\begin{itemize}
\item The goal of this work is to provide a fault injection platform flexible and optimized for
FPGA-based systems, allowing emulate configuration faults on SRAM-based FPGAs

\item Define CR environment in the context of future aircraft structures  at the level of electrical systems
\item Study existing databases of effects of CR at electrical systems level
\item Complete the analyses of the result of the CR characteristics recorded and derive the consequences in the aircraft embedded system
\item Develop the computer model of the CR effects
\item Simulate numerically the effect at component and electrical system level
\item Develop a strategy for evaluating the robustness of systems against CR
\item Propose update of CR requirements for electrical systems
\item Proposed the methodology and the models based on data observed in on-board experiments


\end{itemize}



The main challenges we foresee are:
\begin{itemize}
\item Development of the complex circuits and testing under radiation, e.g., soft-error analysis for sequential circuits
\item Make a model at higher-level of abstraction from the data extracted at lower level 
\item Experimental set-up for bombardment

\end{itemize}


  

\section{Novelty and Impact}


\begin{itemize}
\item The development and implementation of an early validation strategy at higher abstraction level helps to identify at what extent mitigation strategies are required
\item Study the system susceptibility under neutron-induced single even effect
\item Compare the neutron induced and proton induced errors
\item Signature for the sequential circuit
\item Computer model to study CR effects at early in the embedded system design
\end{itemize}


%%% Local Variables:
%%% mode: latex
%%% TeX-master: "../Document"
%%% End:
       % Introduction au sujet de recherche.
\include{./sections/Related}  % Revue de littérature.
\Chapter{PROPOSED APPROACH}\label{sec:approach}\selectlanguage{english}

This chapter is dedicated to the methodology that we propose to exert to achieve the objectives of this research project, i.e. Methodology and Algorithms for High-level Modelling of Cosmic Radiations Impacts on Electrical Systems. First of all, we will identify four main research axes: (1)  Fault emulation platform for sequential circuits to generate signatures; (2) radiation-based experiments; (3) high-level modeling to study radiation impacts on electrical systems; and (4) Simulator- isoneo.

In addition to the development of our research along these axes, we also have a plan to implement a technology demonstrator on FPGA and, fly in an aircraft, e.g., \textbf{CMC BEE platform}.

\section{Research Axis 1: Measuring Sequential Circuit Testing and Reliability}

\subsection{Emulation Environment and Framework}
\subsection{Sequential Circuit Fault Injection Mechanism}
\subsection{Sequential Circuit Test Generation}


\section{Research Axis 2: Modeling of Sequential Circuit}
\subsection{Modeling of Sequential Circuit with Model Checking}

\subsection{Modeling of Sequential Circuit with BDD/ADD}

\subsection{Modeling of Sequential Circuit with Monte-carlo Techniques}

\subsection{Modeling of Sequential Circuit with Markovian-Chain Analysis}



%The fault injection platform proposes for this project emulates SEUs, more specifically single-bit upsets (SBUs) within the configuration memory of SRAM-based FPGAs. We will study the effects of SEU on sequential circuits, and introduce a  framework for analyzing and detecting them. We will do the modeling and analysis of sequential circuits susceptibility to soft errors. Accurate sequential SEU estimation requires capturing the mechanism of error propagation and masking at both combinational and sequential levels. The challenging task for the sequential circuits under SEUs: the difference between sequential and combinational circuits from the context of ATPG and single stuck-at fault model. In this project, we will concentrate on sequential synchronous circuits. The problem we will face and encounter that the controllability of auxiliary inputs and observability of secondary outputs for the sequential circuits.  We will study and implement a technique which eases sequential circuits testing and ATPG by making controllability and observability much simple.
%
%\subsection{Fault Injection}
%
%We need to examine the behavior of a design under faults. For fault injection there are two strategies: (a) \textit{Software based fault injection}, and (b) \textit{FPGA based fault injection}. 
%
%\begin{itemize}
%
%\item Two methods will develop for software based fault injection. First, the source HDL code is modified to allow fault injection. Second, the simulation tool is used to force the error injection during simulation. 
%\item For FPGA-based simulation, we will use single error mitigation core from Xilinx~\cite{xilinx}. The idea is to integrate this core with our system to generate a modified bitstreams to emulate the occurrence of errors. We will use this strategy.
%\end{itemize}





\section{Research Axis 3: Radiation Bombardment}
This part of the project is a neutron-induced Single Event Effect test in a commercial FPGA from Xilinx. The primary objective is to investigate the radiation effects reliability for the critical application. We will implement the sequential circuit and data acquisition system. The results we want to achieve to drive signatures for the sequential circuits. Our focus is on the analyzing the impact of multiple errors in state flip-flops, during the cycles following the cycle when faults occur. The following milestones we want to achieve from radiation bombardment experiment.

\begin{itemize}
\item Modeling of SEU, MBU and analyzing their effect on logic circuits.
\item Evaluation of changes in error rates due to SEUs in sequential circuits.
\item Compute the error probability, and signatures from bit-upsets can vary for different outputs and different circuits. 
\item Evaluation of the impact of multiple flip-flop upsets in sequential circuits.
\item Determining the outputs that are most susceptible to errors due to faults in logic.
\item Determining the parts of the circuit (gates or gate clusters) that have the largest impact on circuit error probability.
\item Estimation of lower and upper bounds of circuit susceptibility to transient.
\end{itemize} 
%
%\section{Research Axis 3: High-level Modelling}
%
%To model and analyze the sequential circuit susceptibility to soft errors, we need to used the approximate methods, e.g.,
%\begin{itemize}
%
%\item Binary Decision Diagram and Algebraic decision diagram.
%\item Markov-chain analysis based error rate estimation, which can provide steady-state Single-Error rate estimates following a hit. 
%\item A  Monte Carlo for SEU Analysis of Sequential Circuits based on the probability of the bit-flips and estimates the states outputs and signatures.
%
%\end{itemize}
%\subsection{Simulator}
%
%We also have a plan to develop a simulator with the help of \textit{isoneo}. The simulator is based on the Matlab / Simulink models. The simulator takes the input a parameterization file corresponding to the operational architecture of the system. This file is generated from the configurator; it is in XML format.
%From this configuration, the simulator initially initializes a model of the system failure tree.
%This model is then exploited dynamically during the simulation phase to evaluate the level of reliability of the system and its components. The simulator executes the simulation model with the constraints and concludes a level of safety for each equipment and the global system.

\section{Optional Research Axis: Fault Mitigation}
Fault-mitigation can be achieved in two ways: preventing faults from happening and
recovering after their occurrence. Fault preventing is achieved by using hardened components and/or shielding. But fault preventative is not a viable solution in terms of a project cost. More complex fault-mitigation methodologies can be implemented at the architectural level. We need to develop some fault-mitigation strategies like triple module redundancy with  dynamic reconfiguration of the hardware~\cite{jacobs2012reconfigurable} and/or something like the work presented in 
%Jacobs \emph{et al.}~\cite{jacobs2012reconfigurable} and Alderighi \emph{et al.}~\cite{violante} promote the use of SRAM-FPGAs for reconfigurable fault-tolerant space applications.
~\cite{jacobs2012reconfigurable} used fault tolerance framework (RFT) that enables system designers to dynamically adjust a system's level of redundancy and fault mitigation based on the varying radiation incurred at different orbital positions. Notably, the reconfigurable fault tolerance framework in~\cite{jacobs2012reconfigurable} is based on an upset rate modeling tool that used to capture time-varying radiation effects in a given orbit.


\section{Project Plan}


\textbf{Summary}

\textbf{Phase  01:} The emulation platform will be the starting point of research. We will use the SEUs for the configuration memory upsets. Selection of a suitable benchmark, which is probably ITC'99~\cite{ITC}used for the testing purpose. We will evaluate the bits sensitivity as well. We will implement the prototype. 

\textbf{Phase  02:} Evaluate the experimental setup under the neutron radiation at Triumf.


\textbf{Phase 03:} Develop an efficient methodology and high-level model for soft-error of sequential circuits, i.e., Monte-Carlo sampling, approximate approaches, symbolic methods for efficient estimation. The simulator development will keep with all these three phases.



%\section{Timetable}
%The development of the tasks identified in Chapter~\ref{sec:approach}, and the most important milestones of this project are presented in Figure~\ref{timetable}.
%In our intentions, the design of a time predictable computer architecture, the development of novel timing analysis techniques, and the FPGA prototypes implementation will unfold as a series of sequential tasks with relatively small interleaving. Dependability and real-time requirements, on the other hand, should be kept in mind throughout the whole advancement of the project.
%
%\begin{figure}[h]
%\centering
%\begin{tikzpicture}
%\begin{ganttchart}[
%x unit=0.36cm,
%y unit title=1.0cm,
%y unit chart=1.5cm,
%%vgrid,
%hgrid,
%inline,
%]{1}{48}
%\gantttitle{Research Project}{48} \\
%\gantttitle{2016}{12} \gantttitle{2017}{12} \gantttitle{2018}{12} \gantttitle{2019}{12}\\
%
%
%
%\ganttbar[bar height=.4]{Literature Review and Comp .Exam}{6}{24}\\
%%\ganttmilestone[]{Comp. Exam}{20}\\
%%\ganttmilestone[]{AHS}{6}
%%\ganttmilestone[]{TODAES}{12}\\
%\ganttbar[bar height=.4]{Emulation Platform for Seq. ckt}{6}{35} \\
%\ganttbar[bar height=.4]{Radiation Experiment}{13}{24} \\
%\ganttbar[bar height=.4]{Modelling and Simulator}{25}{42} \\
%%\ganttmilestone[]{TAAS}{30}\\
%%\ganttbar[bar height=.4]{Novel Timing analysis techniques}{22}{38} \\
%%\ganttmilestone[]{DAC}{36}\\
%%\ganttbar[bar height=.4]{FPGA Implementation}{27}{38} \\
%%\ganttmilestone[]{TRETS}{40}\\
%\ganttbar[bar height=.4]{Tech. Demo.}{35}{42} \\
%%\ganttmilestone[]{IAC}{44}\\
%\ganttbar[bar height=.4]{Thesis Writing}{35}{46}
%%\ganttmilestone[]{Thesis}{44}\\
%%\ganttbar[bar height=.4]{The \emph{PolyOrbite} Project}{1}{44} 
%
%%\ganttlink{elem0}{elem1}
%%\ganttlink{elem0}{elem2}
%%\ganttlink{elem0}{elem3}
%
%%\ganttlink{elem3}{elem4}
%
%%\ganttlink{elem5}{elem6}
%%\ganttlink{elem5}{elem7}
%
%%\ganttlink{elem7}{elem8}
%\end{ganttchart}
%\end{tikzpicture}
%\caption{Timetable.}
%\label{timetable}
%\end{figure}









         
\include{./sections/Preliminary}         
\include{./sections/Conclusions}         % Conclusion.
%\backmatter
%\renewcommand\bibname{RÉFÉRENCES}
\renewcommand\bibname{REFERENCES}
\bibliography{Document}
\bibliographystyle{polymtl}  % Format de la bibliographie.
%
%\ifthenelse{\equal{\AnnexesPresentes}{O}}{
%\appendix%
%\newcommand{\Annexe}[1]{\annexe{#1}\setcounter{figure}{0}\setcounter{table}{0}\setcounter{footnote}{0}}%
%\include{/sections/Annexes}}{}
\end{document}
